\pagenumbering{gobble}% Remove page numbers (and reset to 1)
\clearpage
\begin{abstract}
    Das Wayfinder Seminar\footnote{Wayfinder (dt. ``Wegfindung‘‘) sei nachfolgend als Kurzform für den absolvierten Workshop verwendet} ist ein an der Stanford University konzipierter Workshop, der unter der Leitung von Klaudia Thal\footnote{https://www.linkedin.com/in/klaudia-thal-17bab710/}, Dr. Claudia Nicolai\footnote{https://www.linkedin.com/in/claudia-nicolai/} und Dr. Martin Schwemmle\footnote{https://www.linkedin.com/in/martin-schwemmle-455322184/} im Sommer 2020 am Hasso-Plattner-Institut durchgeführt wurde. Ziel des Kurses war es, den Teilnehmenden bei der Erstellung ihrer Zukunftspläne und der allgemeinen Wegfindung zu helfen.
    Der Workshop fand aufgrund der gegenwärtigen Situation im remote Format statt. Daher mussten vor allem für das Prototypisieren der Zukunftspläne neue Konzepte entworfen werden. Hierbei setzte das Team der HPI DSchool auf die Methode Lego\textsuperscript{\textregistered} Serious Play\textsuperscript{\textregistered}\footnote{https://www.lego.com/en-us/seriousplay}\footnote{Auf die nachfolgende Kennzeichnung der registrierten Marke wurde aus Gründen der Lesbarkeit verzichtet. }.


    Diese Methode wird auf den folgenden Seiten im Kontext des Prototypings eingeführt und mit anderen Methoden verglichen. Darauf folgt eine Dokumentation der Anwendung dieser Methode im Wayfinder. Abschließend werden alternative Integrationsformen vorgestellt und die Anwendung der Methode in einer Diskussion abgeschlossen.

\end{abstract}