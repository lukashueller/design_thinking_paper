\section{Motivation}

Das Prototyping\footnote{Prototyp (lat. prototypon = Urbild) sei nachfolgend maskulin dekliniert} ist die fünfte und damit vorletzte Phase im Design Thinking Prozess nach (Weinberg \#todo). Diese Phase erfährt eine besondere Wichtigkeit, da hierbei die entstandenen Lösungsvorschläge und Ideen der Phasen zuvor erstmals zum Leben erweckt werden. (Abbildung ~\ref{fig:dt_process} auf Seite ~\pageref{fig:dt_process})

Für mich persönlich nimmt diese Phase des Design Thinking Prozesses eine besondere Rolle ein. Vor allem in kurzen Workshops (DT Sprints), können die Teilnehmer in schnell ein anfassbares Ergebnis erschaffen. In dieser Phase erwachen die zuvor theoretisch betrachteten Ideen und Lösungsvorschläge zum Leben und werden visuell präsentierbar. Die Methoden des Prototypings sind unkompliziert und sofort anwendbar. Sie bedürfen nur einer kurzen Erklärung und können ohne großen finanziellen Aufwand umgesetzt werden.

Das Prototyping geriet für mich stärker in den Fokus als ich erfuhr, dass das Wayfinder Seminar remote stattfinden wird. Die Umsetzung des Prototypings in einem remote Workshop ist im Vergleich zu den anderen Phasen des Design Thinking Sprints deutlich schwieriger. Daher wollte ich insbesondere erfahren, wie die Experten der HPI D-School\footnote{https://hpi.de/school-of-design-thinking.html} diese Phase im Remote-Setup unterbringen. 

Die Methode des Lego Serious Play (LSP) interessierte mich schon länger. Eines der größten Ziele in meinen eigenen Workshops ist es, allen Teilnehmenden der Sessions dieselben Ausgangsbedingungen zu bieten. Dies ist technisch grundsätzlich möglich, jedoch fallen Teams gerade beim Prototyping oftmals wieder in gewohnte Organisationsstrukturen zurück. Dieses Verhaltensmuster erkennt LSP und sorgt aufgrund der eigenständigen Arbeit des Individuums dafür, dass alle Teilnehmenden in gleicher Art und Weise kreativ sein können.
