\section{Diskussion}
\label{diskussion}

\subsection{Diskussion der Anwendung von LSP im Wayfinder}
Die Methode Lego Serious Play wurde für den digital stattfindenden Wayfinder-Workshop als wesentliche Prototyping-Methode ausgewählt. Diese Entscheidung ist aus meiner Sicht sehr überlegt getroffen worden.
Bei einer On-Site-Variante des Workshops hätten sich sicherlich zahlreiche Prototyping-Methoden angeboten. Aufgrund des remote Setups entfallen jedoch ziemlich alle der unter \ref{intro_prototyping} beschriebenen Methoden aufgrund des benötigten Materials. Für Skizzen oder Papier-Prototypen haben nicht alle Teilnehmenden das benötigte Kreativequipment zwangsweise vor Ort. In den Räumen der HPI DSchool gibt es hierfür gemeinsame Materialiensammlungen, sodass allen Teilnehmenden dieselben Mittel zur Verfügung stehen. Das kann bei einem nicht vor Ort stattfindenden Workshop nicht gewährleistet werden, weshalb diese Methoden eher nicht zu empfehlen sind. Rollenspiele eignen sich für individuelle Prototypen einer Person wie sie im Wayfinder angefertigt werden nicht so gut. Videoclips sind meinem Ermessen nach zu aufwändig für dieses Seminar.

Betrachten wir nun die aufgrund der speziellen Situation ausgewählte Methode. Bereits mit einem standardisierten Beutel an Legosteinen ist jeder Teilnehmende bereit für diesen Workshop. Diese Materialien bekamen wir postalisch zugesendet. Die weiße Unterlage war ebenso in dem Paket enthalten. Den zum Vorstellen benötigten Stift sollte jeder selbst beisteuern können. Somit hatten alle Teilnehmenden dieselben Ausgangsbedingungen für die Prototyping-Session.

Ein Nachteil von LSP bei On-Site Workshops kann aus meiner eigenen Erfahrung steigender Druck im Team sein. So lassen sich Teilnehmende während der Arbeit an ihrem individuellen Modell durch den Fortschritt der anderen oder den u. U. hohen Geräuschpegel während des Protoypings unter Druck setzen. Hierbei kam uns das remote-Format entgegen, welches dafür sorgte, dass alle Teilnehmenden akustisch ungestört und auch ohne den direkten Druck von Tischnachbarn an ihrem Modell arbeiten konnten.
Ebenso benötigt das Prototyping mit LSP nur sehr wenig Platz. Das Zusammenbauen des Modells mit maximal A4-Blatt-Größe ist auf dem heimischen Schreibtisch während einer Videokonferenz problemlos möglich gewesen. 

Um das Wayfinder-Seminar abwechslungsreicher zu gestalten, hätte man durchaus auch eine weitere Prototyping-Methode verwenden können. Hierfür wäre - wie oben beschrieben - lediglich das Zeichnen und Skizzieren geeignet gewesen. Aufgrund der benötigten ``grundlegenden Kreativität‘‘ für diese Methode, eignet sie sich bei unerfahrenem Teilnehmerfeld jedoch nicht so gut. Wenn Teilnehmer beim Prototypisieren nicht weiterkommen, kann ihnen bei remote Workshops nicht so schnell geholfen werden. Auf diese Hilfe ist man bei Methoden, die eine höhere Eigenkrativität erfordern, häufiger angewiesen.

Weitere Nachteile von Lego Serious Play sollen an dieser Stelle kritisch hinterfragt werden. Für einen zielführenden Workshop ist es aus meiner Sicht zwingende Voraussetzung, dass ausgebildete LSP ``Vermittler'' (Fachjargon: ``Facilitator'') durch den Workshop leiten. Ebenso sollten die betreuenden Coaches mit der Methode vertraut sein. Die benötigten Fortbildungsseminare sind jedoch recht kostenintensiv. Obwohl der  Zugang zur Methode sehr einfach ist, kann der spielerisch-kreative Ansatz von LSP bei einigen Teilnehmenden als nicht wissenschaftlich empfunden und damit die Methode unterschätzt werden. Selten sorgt auch der Name für einige Vorbehalte. Den Teilnehmenden mit dieser Grundeinstellung fällt der Zugang zur Methode vergleichsweise schwer. In unserem Workshop konnte ich das kaum beobachten. Das liegt aus meiner Sicht daran, dass uns die Coaches vor dem ersten Prototyping mittels kleiner Aufwärmübungen in das richtige Mindset gebracht haben, um mit der Methode zu arbeiten.

Erwähnt werden sollten noch die verhältnismäßig hohen Anschaffungskosten der LSP-Sets. Die so genannte ``Window Exploration Bag'' wurde in unserem Workshop verwendet. Diese kleinen Beutel eignen sich vor allem für kurze Workshops und kosten pro Person ca. 3.50 €\footnote{https://www.lego.com/de-de/product/window-exploration-bag-2000409}. Für längere Workshops eignen sich jedoch Sets mit verschiedenen Sonderbausteinen. Die hierfür vorgesehenen Sets von LEGO® kosten 24 € pro Person\footnote{https://www.lego.com/de-de/product/starter-kit-2000414}. 

Zudem erfordert die Anwendung von LSP bei Vor-Ort-Workshops eine besondere Aufmerksamkeit. Für ein entspanntes und kreatives Arbeiten sollte jedem Teilnehmenden ausreichend Arbeitsraum zur Verfügung gestellt werden, um konstruktiv an seinem Modell zu arbeiten. Der benötigte Platz ist somit größer als z. B. bei der Methode ``Sketching'' (Zeichnen). Darüber hinaus entsteht beim Durchsuchen von Lego-Boxen oder dem Umherschieben von Steinen ein nicht zu unterschätzender Geräuschpegel.

\pagebreak

\subsection{Fazit}

Die Methode Lego Serious Play ist sehr leicht und schnell erlernt. Gerade in remote Workshops eignet sie sich hervorragend zum Prototypisieren. Speziell auf dem Gebiet des individuellen Prototypings, wie es beim Wayfinder notwendig ist, ist sie meiner Meinung nach das Mittel der Wahl. Die oben genannten Nachteile von LSP wurden durch die Profis der HPI D-School gekonnt umgangen. Somit ergab sich ein sehr runder Workshop mit einem guten Exkurs in diese Methode!