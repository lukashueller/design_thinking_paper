\section{Implementierung von LSP im Wayfinder Seminar}

\subsection{konkrete Anwendung von LSP}
\label{implementation}

Im Wayfinder wurde die LSP Methode zum Prototypisieren unserer Zukunftspläne genutzt. 

In der ersten Woche des Seminars analysierten wir unsere langfristigen Lebensziele. Dazu untersuchten wir das eigene Weltbild und formulierten es aus. Dies half dabei, unsere Lebensziele für in 20-30 Jahren zu definieren. Zudem lernten wir in der ersten Woche unsere Triade (ein festes Dreierteam für den Rest des Workshops) kennen. Da bis zu diesem Tag nicht alle Teilnehmenden das LSP Set erhalten hatten, konnte die Methode in der ersten Woche noch nicht eingesetzt werden. 

In der zweiten Woche betrachteten wir dann unsere Ziele für die kommenden 5 Jahre etwas genauer. Dabei gingen wir erstmals von unserer aktuellen Situation aus und überlegten, wo es Überschneidungen mit den Lebenszielen gibt. Diese fassten wir auf dem so genannten ``Coherence-Board‘‘ zusammen. Zum Prototypisieren der Lebensziele wurde dann die LSP-Methode wie folgt angewendet. Anfangs stiegen wir mit kurzen Übungen ein, um uns mit den Steinen und der Methode vertraut zu machen. Im Anschluss daran bauten wir das Modell ``My 2020‘‘. Dieses Modell sollte ausdrücken, welche Ziele oder Wünsche noch in diesem Jahr in Erfüllung gehen sollen. In meinem Fall beschrieb es vor allem den Einklang von Studium, Beruf und Zeit für Urlaub sowie Privatleben. Ein Foto des entstandenen Prototyps ist im Anhang (Abbildung \ref{fig:week2}) beigefügt.

Das Motto der dritten Wayfinder-Einheit lautete ``Act now''. Alle Teilnehmenden entwickelte einen konkreten Plan, wie sie in den kommenden Monaten ihr Leben verbessern wollen. Auch diese entstandene Ideensammlung wurde am Ende wieder mittels LSP prototypisiert. Aufgrund des Feedbacks, dass mein erstes Modell (siehe Abbindung \ref{fig:week3}) zu eingeengt und starr wirkt, habe ich mich für ein modulareres Modell entschieden. Ich versuchte mit einzelnen Bausteinen einen Ort abzubilden, an dem ich sowohl das Arbeiten als auch die Entspannung und den Urlaub vereinen kann. Die Blumen im oberen Bereich des Modells symbolisieren, dass dieser Ort nicht die Großstadt ist, in der ich momentan lebe. Der Propeller soll darstellen, dass durchgehend die Möglichkeit besteht, wieder nach Hause zurückzukehren oder den Ort zu wechseln.


In der letzten Woche nutzten wir die LEGO-Bausteine zum Erstellen unseres finalen Wayfinder-Modells (siehe Abbildung \ref{fig:week4}). Hierbei versuchte ich alle Säulen, die meine Zukunftspläne beeinflussen, in einem Modell abzubilden. Daher entschied ich mich, einen 4-zackigen Stern zu bauen, welcher gleichzeitig als Wippe fungiert. Somit soll ausgedrückt werden, dass wenn sich die Person einem Thema widmet, die anderen Säulen sprichwörtlich ``In der Luft hängen‘‘. Die vier Säulen meines Zukunftsmodells setzen sich aus Urlaub, dem Leben in der Heimat mit Eltern und Freunden, dem Studium und dem beruflichen Alltag zusammen. Die Balance zwischen diesen Punkten wird durch meinen finalen Prototyp abgebildet. \newline

Besonders interessant war für mich, dass wir jede Woche unter völlig neuen Gesichtspunkten begannen, den Prototypen zu erbauen. Dennoch kam am Ende jedes Prototypings ein Modell heraus, welches dieselben vier Säulen wiederspiegelte. Die Art der Anordnung und die damit verbundenen Hintergrundgedanken änderten sich jedoch bei jeder Iteration. Aufgrund des Feedbacks, dass das erste Modell so einengend wirkt, wollte ich in der dritten Woche ein flexibles Modell, bei dem nicht alle Teile untereinander verbunden sind, konstruieren. Jedoch merkte ich, dass ich genau diesen Zusammenhang der Säulen benötige, um mich selbst von der Realisierbarkeit der Zukunftspläne zu überzeugen. Daher landete ich zuletzt mit dem Stern bei einem offenen Modell, bei dem alle Teile miteinander verbunden sind.

\subsection{andere Integrationsformen der Methode}

Wie im Abschnitt \ref{implementation} beschrieben, wurde die LSP-Methode mehrfach im Kurs angewendet. Auf der Suche nach anderen Implementierungsmöglichkeiten bin ich aufgrund der guten Anwendung (siehe dazu auch Abschnitt \ref{diskussion} auf Seite \pageref{diskussion}) im Seminar und dem ungewöhnlichen remote Setup lediglich auf Verbesserungsmöglichkeiten gestoßen.
Bei einer Vor-Ort-Variante des Workshops hätte es sich sicherlich angeboten, das Vorstellen der Prototypen in noch kleineren Gruppen vorzunehmen. Somit hätte man unter Umständen ein direkteres und umfangreicheres Feedback zum eigenen Prototyp erhalten.

Darüber hinaus würde ich die Methode LSP für ein morgendliches Warmup verwenden. Während der remote Workshops wurde täglich eine ``Zeit zum Einchecken‘‘ von 30 Minuten gewährt. Für diese Zeit wären eine oder zwei kleine Aufgaben mit den Lego-Sets denkbar, mit der die Teilnehmenden das richtige Mindset erlangen, um in den Tag zu starten. Hierbei würden sich Modelle zur aktuellen Gefühlslage oder der Zufriedenheit mit der Hausaufgabe gut zum Bauen eignen. Genau diese Themen wurden zu Beginn eines jeden Workshops mit allen Teilnehmenden besprochen.