\section{Status Quo}\label{sec:statusQuo}

\subsection{Introduction to Design Thinking}

Design Thinking is an iterative creative process using methods that deliver user-oriented solutions to a wide range of problems. Particular emphasis is placed on the creative and artistic approach, which is already anchored in the name "Design Thinking". The non-linear, iterative core process includes six phases. In addition, further phases (such as the implementation of the solution) can be added.

The first three phases UNDERSTAND, OBSERVE, and DEFINE deal with the problem space analysis. Hereby the given problem space is explored and based on this the point of view is defined. These three phases serve the basic understanding and analysis of the real underlying problem. They should not yet consciously allow thoughts for possible problem solutions. In the following fourth step, possible solutions to the previously defined problem are brainstormed. This is followed by the PROTOTYPE phase which is described in detail in this paper. In this fifth phase of the overall process, the ideas are brought to life and made tangible and presentable using various methods. Based on the prototype created in this phase, the results can be evaluated in the final TESTING phase. 

The iterative process structure given by the model allows switching between different phases (See figure~\ref{fig:dt_process} on page~\pageref{fig:dt_process}). In this way, ideas that have already been developed can be made more concrete, prototypes can be modified or problem solutions can be realigned. Therefore this process is considered non-linear and especially creative.

\subsection{Introduction to Prototyping}

The phase of Prototyping will be considered in more detail below. 

The starting point are the developed ideas based on the described problem. The goal of prototyping is to use different methods to make the previously generated ideas tangible. For this purpose, the project team develops a prototype that reflects a proposed solution for the problem. This prototype is often touchable and helps the creators to find ideas and to self-reflect the solution. Since the first prototype rarely covers all customer requirements, it is often iterated over it for improvement. The prototype helps in the exchange with the customer to present him a solution for his problem. Prototyping is characterized by a comparatively low expenditure of work and time and relatively low costs.
In order to create a meaningful prototype, there are various methods in Design Thinking (taken from: \#todo). These are now presented and briefly evaluated: \\

\textbf{Sketches and drawings:}	This method requires only paper and some pens. Simple sketches should help to illustrate the abstract solution. The resulting illustrations are well suited for exchange in a team - especially in early iteration phases. Sketches and mind maps can be used as a basis for further iterations. It should be noted, that those participants who are not so familiar with sketching, drawing, or wireframing may not be able to contribute creatively enough.

\textbf{Paper Prototype:} This method is suitable for sketching digital products such as apps or web pages. It involves drawing the mockups on paper and gluing them together to create a prototype. Web applications can also be made tangible on paper, for example by folding over paper windows. The Paper Prototype is an easy to learn and inexpensive method analogous to simple sketching. All you need are a few sheets of paper and basic craft materials such as glue, scissors, and pens.

\textbf{Role Play:}	Here the focus is on social interaction. If the problem has a social background, the solution can be presented in a role play. In the interaction of the actors, the problems, and their possible solutions for the socio-political issues become clear. The presentation of a socio-political problem requires a strong abstraction when visualizing with a paper model. This is not the case with a role play.
In principle, the method does not require any materials; however, it is recommended to work with some requisites. In order to support the emotional situation or the embodied background of the actor, the HPI DSchool uses wigs or clothing for this purpose.

\textbf{Video Clip:} The goal of the method is to present the solution in the form of a short video. This method differs from the others in that the film does not need any further explanation. The resulting clips can thus be used as a marketing video, for example. Since the final product of the prototyping is digitally available in this case, the video can be made available to possible stakeholders quickly and easily. In the videos, the product idea is often explained by means of explanatory sketches. Video clips are often only used in the final iterations of the prototyping phase due to the high creation effort. For the creation of meaningful media prototypes, the required prior knowledge should not be underestimated. The conception and creation of a video is considered relatively challenging among the methods. 

\newpage

\subsection{Introduction to Lego Serious Play}