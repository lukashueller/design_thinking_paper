\documentclass[
	DIV=9,						% Einstellung für den Seitenaufbau (Standardwert)
	bibliography=totoc, 		% Literaturverzeichnis ins Inhaltsverzeichnis
	listof=totoc,				% Abbildungs- und Tabellenverzeichnis ins Inhaltsverzeichnis
	abstract=true,				% Mit Abstract
	11pt,						% Schriftgröße
]{scrartcl} 					% scrartcl für kurze Arbeiten (Achtung: es steht kein \chapter zur Verfügung)
%]{scrreprt} 
%saubere Zeilenumbrüche in URLs
\usepackage[hyphens]{url}
\sloppy

% Euro-Symbol
\usepackage{eurosym}
\usepackage{smartdiagram} %für Smart-Art zur SHA

\usepackage[utf8]{inputenc}		% Eingabedaten liegen im UTF8-Format vor
\usepackage[T1]{fontenc} 		% Ausgabezeichensatz
\usepackage{lmodern}			% Zeichensätze für die Arbeit
% -------------------------------------------------------------------------------
\usepackage[fleqn,				% Mathematische Formeln linksbündig eingerückt
			%leqno,				% Gleichungsnummerierung links
			intlimits,			% Integralgrenzen oben und unten
			]{amsmath}			% Mathematik
% -------------------------------------------------------------------------------
\usepackage[ngerman, english]	% Auswahl der Sprachen, für die Layout- und Umbruchregeln bekannt sein sollen
					{babel}		% Die Sprachen werden mit \selectlanguage{} ausgewählt. Die aktuell Wahl ist ngerman


\usepackage{graphicx}			% Einbinden von Grafiken
								% da die Kompilierung über 'pdflatex' erfolgt, stehen '.jpg', '.png' und '.pdf'
								% als Grafikformate zur Verfügung
%\usepackage{textcomp}  %für DollarZeichen
% Für Beispiel
\usepackage{blindtext}


% Paket aus den KOMA-Klassen für die Einstellungen der Kopf- und Fußzeilen
% Nachfolger von scrpage2
% Eine Alternative wäre das Paket 'fancyhdr'
% Siehe dazu Kapitel 5 im SCRGUIDE
\usepackage{scrlayer-scrpage}
\automark{section}			% \automark -> Kapitel 5.5 im SCRGUIDE
\clearscrheadfoot			% Kopf- und Fußzeilen leeren
\ohead[]{\headmark}			% \ohead -> Abbildung 5.1 im SCRGUIDE
                            % \headmark -> Kapitel 5.5 im SCRGUIDE
% \ihead{\includegraphics[height=\baselineskip]{img/hpi_logo.png}}
                            % \ihead -> Abbildung 5.1 im SCRGUIDE
\ofoot[\pagemark]{\pagemark}			% \ofoot -> Abbildung 5.2 im SCRGUIDE
                            % \pagemark -> Kapitel 5.5 im SCRGUIDE
\setkomafont{pagehead}{\scshape} % Siehe Beispiel auf Seite 237 im SCRGUIDE
\addtokomafont{pagehead}{\bfseries} % Befehle \setkomafont und \addtokomafont -> Kapitel 3.6 im SCRGUIDE
%\addtokomafont{pagehead}{\color{green}} % Befehle \setkomafont und
% \addtokomafont -> Kapitel 3.6 im SCRGUIDE \addtokomafont{section}{\color{red}} % Befehle \setkomafont und \addtokomafont
% -> Kapitel 3.6 im SCRGUIDE
\pagestyle{scrheadings}
\setlength{\headheight}{24pt}
\setlength{\footheight}{24pt}


% Wasserzeichen auf dem Dokument -----------------------------------------------------------------
%	Das ist dafür gedacht, dass man Vorabexemplare herausgeben kann, ohne dass sie später
%	als Originale herumgereicht werden.
% \usepackage{draftwatermark}							% Wasserzeichen immer in grau
% \SetWatermarkLightness{.95}							% Grauwert, zwischen 1 (weiß) und 0 (Schwarz)
% \SetWatermarkAngle{60}								% Winkel, um den das Wasserzeichen gedreht worden ist.
% \SetWatermarkFontSize{5cm}							% Schriftgröße beim Wasserzeichen (max 5cm)
% \SetWatermarkScale{.5}								% nachträgliche Skalierung
% \SetWatermarkText{\sffamily\textbf{Stand: \today}}	% Wasserzeichentext

\usepackage[margin=1in]{geometry} 
% \usepackage[left=3cm,right=2.5cm,top=2.5cm,bottom=2.5cm]{geometry} 
\usepackage{xcolor}
\usepackage[]{todonotes}

% Korrekte Einheitendarstellung -------------------------------------------------------------
\usepackage[output-decimal-marker={,}, 	% Dezimal-Trennzeichen ist in Deutschland das Komma
			]{siunitx}


\usepackage{array} % einfache Erweiterungen für die Tabellenformatierungen
% ---------------------------------------------------------------------------------
% Wenn man unbedingt den Zeilenabstand erhöhen muss, dann über das Paket setspace
\usepackage{setspace}
\onehalfspacing

\usepackage{paralist}%für Aufzählungsumgebungen

%Abbildungen
\usepackage{caption}
\usepackage{subcaption}

% Abkürzungverzeichnis
\usepackage[printonlyused	% nur die Abkürzungen aufführen, die auch verwendet werden
			]{acronym}


\usepackage[citestyle=authoryear,backend=biber,style=authoryear]{biblatex}
\bibliography{lit.bib}

% Nützliche Pakete
\usepackage{csquotes}		% Anführungsstriche
%\usepackage{relsize}		% Wenn man Zwischengrößen in den Schriftgrößen benötigt, bitte die Doku aufmerksam lesen
%\usepackage{listings}		% Für die Behandlung von Quellcode
%\usepackage{media9}		% Einbinden von Videos
%\usepackage{animations}	% Eigene Animationen
%\usepackage{attachfile2}	% Einbau von Dateien in die pdf-Datei
%\usepackage{xcolor}		% Damit man mit Farben umgehen kann, wird aber von vielen Paketen mit geladen (z.B. hyperref)
\usepackage{makeidx}\makeindex % Für ein Stichwortverzeichnis (es gibt auch eine Alternative namens xindy - glaube ich)
%\usepackage{titlesec}
%\makeatletter
%\renewcommand\paragraph{\@startsection{paragraph}{4}{\z@}%
%	{-2.5ex\@plus -1ex \@minus -.25ex}%
%	{1.25ex \@plus .25ex}%
%	{\normalfont\normalsize\bfseries}}
%\makeatother
\setcounter{secnumdepth}{3} % how many sectioning levels to assign numbers to
\setcounter{tocdepth}{3}    % how many sectioning levels to show in ToC
% für paragraph als subsubsubsection
% Für sehr interessierte und engagierte Leute ;)
\usepackage{tikz}			% Grafiken mit LaTeX - extrem mächtig und schön (Alternative wäre psstricks)
\usetikzlibrary{arrows,shapes,positioning,shadows,trees}
\usetikzlibrary{shadings}
\usepackage{smartdiagram} %smart art diagramme
\usepackage{adjustbox}
\usepackage[final]{pdfpages} %für includepdf
%\usetikzlibrary{shapes,snakes}
%\tikzstyle{every node}=[draw=black,thin,anchor=west, minimum height=2.5em]
% \usepackage{pgfplots}		% setzt auf TikZ auf (lädt es also) und ist speziel für die Datenvisualisierung
% \usepackage{ifthen}		% (La)TeX bietet schon selber if-then-Möglichkeiten, durch das Paket werden sie angenehmer im Handling
% \usepackage{forloop}		% auch Schleifen sind möglich

% Laut Dokumentation sollte diese Paket als Letztes eingebunden werden.
\usepackage[hidelinks,						% Links im pdf-Dokument nicht farblich hervorheben (gilt nur für die Darstellung auf dem Bildschirm)
			pdfauthor={Tobias Wuttke}, % Schreibt den Autor in die PDF-Datei
			]{hyperref}

\newcounter{speichereRoemischeZahl}
\def\mytitle{The effects of prototyping and Lego Serious Play \\ in leadership seminars} 

\usepackage{graphicx}
\usepackage{lscape}

% %Komma zwischen Fußnoten, gibt leider eine Warnung
% \usepackage[multiple]{footmisc}

%Seiten im Querformat (große Tabellen die es vermutlich gibt)
%\usepackage{pdflscape}
 
%Tabellen skalieren
%\usepackage{tabularx}
%\usepackage{tabu} %rowfont 
%\usepackage{longtable} %endhead
 
\newlength{\drop}

\usepackage{setspace}
\doublespacing

\begin{document}
    %\listoftodos
	\pagestyle{plain}
	\begin{titlepage}
    \drop=0.1\textheight
    \centering
    \vspace*{\baselineskip}
    \rule{\textwidth}{1.6pt}\vspace*{-\baselineskip}\vspace*{2pt}
    \rule{\textwidth}{0.4pt}\\%[\baselineskip] ausgeklammert am 29.09.2016
    {\Huge WAYFINDER\\
    \vspace*{12pt} %Geandert von 29.09.2016
    \LARGE \mytitle}\\[0.2\baselineskip]
    \rule{\textwidth}{0.4pt}\vspace*{-\baselineskip}\vspace{3.2pt}
    \rule{\textwidth}{1.6pt}\\[\baselineskip]
    \scshape
    Berlin, \today\par
    \vspace*{2\baselineskip}
    Erstellt durch: \\[\baselineskip]
    {\Large Lukas Hüller (796454)\par}\bigbreak
    Betreut durch:\bigbreak 
    {\Large Klaudia Thal \\ Dr. Claudia Nicolai \\ Dr. Martin Schwemmle\par}\bigbreak\bigbreak
    \vfill
    \includegraphics[width=0.5\textwidth]{img/hpi_dschool_logo.jpg}
  \end{titlepage}
    \pagenumbering{gobble}% Remove page numbers (and reset to 1)
\clearpage
\begin{abstract}
\blindtext
\end{abstract}
	\pagestyle{scrheadings} 
	\pagenumbering{Roman}
	\tableofcontents
	\newpage
	\setcounter{speichereRoemischeZahl}{\value{page}}
	\pagenumbering{arabic}
	\section{Motivation}

Das Prototyping\footnote{Prototyp (lat. prototypon = Urbild) sei nachfolgend maskulin dekliniert} ist die fünfte und damit vorletzte Phase im Design Thinking Prozess nach (Weinberg \#todo). Diese Phase erfährt eine besondere Wichtigkeit, da hierbei die entstandenen Lösungsvorschläge und Ideen der Phasen zuvor erstmals zum Leben erweckt werden. (Abbildung ~\ref{fig:dt_process} auf Seite ~\pageref{fig:dt_process})

Für mich persönlich nimmt diese Phase des Design Thinking Prozesses eine besondere Rolle ein. Vor allem in kurzen Workshops (DT Sprints), können die Teilnehmer in schnell ein anfassbares Ergebnis erschaffen. In dieser Phase erwachen die zuvor theoretisch betrachteten Ideen und Lösungsvorschläge zum Leben und werden visuell präsentierbar. Die Methoden des Prototypings sind unkompliziert und sofort anwendbar. Sie bedürfen nur einer kurzen Erklärung und können ohne großen finanziellen Aufwand umgesetzt werden.

Das Prototyping geriet für mich stärker in den Fokus als ich erfuhr, dass das Wayfinder Seminar remote stattfinden wird. Die Umsetzung des Prototypings in einem remote Workshop ist im Vergleich zu den anderen Phasen des Design Thinking Sprints deutlich schwieriger. Daher wollte ich insbesondere erfahren, wie die Experten der HPI D-School\footnote{https://hpi.de/school-of-design-thinking.html} diese Phase im Remote-Setup unterbringen. 

Die Methode des Lego Serious Play (LSP) interessierte mich schon länger. Eines der größten Ziele in meinen eigenen Workshops ist es, allen Teilnehmenden der Sessions dieselben Ausgangsbedingungen zu bieten. Dies ist technisch grundsätzlich möglich, jedoch fallen Teams gerade beim Prototyping oftmals wieder in gewohnte Organisationsstrukturen zurück. Dieses Verhaltensmuster erkennt LSP und sorgt aufgrund der eigenständigen Arbeit des Individuums dafür, dass alle Teilnehmenden in gleicher Art und Weise kreativ sein können.
\newpage
	\section{Status Quo}\label{sec:statusQuo}

\subsection{Einführung in Design Thinking}

Design Thinking ist ein iterativer Kreativprozess, dessen Methoden nutzerorientierte Lösungen für verschiedenste Problemstellungen liefern. Besonders im Vordergrund steht dabei die kreative und künstlerische Vorgehensweise, die bereits im Namen \textit{Design Thinking} verankert ist. Der nicht-lineare, iterative Kernprozess umfasst sechs Phasen. Darüber hinaus können weitere Phasen (wie z. B. die Implementierung der Lösung) hinzugezählt werden.

Die ersten drei Phasen UNDERSTAND, OBSERVE und DEFINE beschäftigen sich mit der Problemraumanalyse. Hierbei wird der gegebene Problemraum erkundet und basierend darauf der Point of view definiert. Diese drei Phasen dienen dem grundlegenden Verständnis und der Analyse des real zugrundeliegenden Problems. Sie sollen bewusst noch keine Gedanken für möglichen Problemlösungen zulassen. Im darauffolgenden vierten Schritt werden mögliche Lösungen zum vorher definierten Problem gebrainstormt. Daran sc  hließt sich die in diesem Paper näher beschriebene Phase PROTOTYPE an. In dieser fünften Phase des Gesamtprozesses werden die Ideen zum Leben erweckt und mittels verschiedener Methoden anfassbar und präsentierbar gemacht. Anhand des in dieser Phase entstandenen Prototyps lassen sich die Ergebnisse in der abschließenden Phase TESTING evaluieren. 


Der vom Modell vorgegebene, iterative Prozessaufbau ermöglicht das Wechseln zwischen verschiedenen Phasen  (Siehe Abbildung ~\ref{fig:dt_process} auf Seite page~\pageref{fig:dt_process}). Damit können bereits entstandene Ideen konkretisiert, Prototypen verändert oder Problemlösungen neu ausgerichtet werden. Deshalb gilt dieser Prozess als nicht-linear und besonders kreativ.
Für eine praktische Erläuterung der Methode empfehle ich das Buch ``Design Thinking Live'' von Prof. Ulrich Weinberg (Weinberg 2015).

\pagebreak
\subsection{Einführung in Prototyping}
\label{intro_prototyping}

Die Phase des Prototypings wird im Folgenden genauer betrachtet.

Ausgangspunkt sind die auf Basis der beschriebenen Problemstellung entwickelten Ideen. Ziel des Prototypings ist es, mithilfe unterschiedlicher Methoden, die zuvor generierten Ideen erlebbar zu machen. Hierzu wird vom Projektteam ein Prototyp entwickelt, welcher einen Lösungsvorschlag für das gestellte Problem wiederspiegelt. Dieser ist oft anfassbar, hilft den Erstellern bei der Ideenfindung und Selbstreflektion der erarbeiteten Lösung. Da der erste Prototyp nur selten alle Kundenwünsche abdeckt, wird zur Verbesserung oftmals über ihn iteriert. Der Prototyp hilft im Austausch mit dem Kunden, ihm eine Lösung für sein gestelltes Problem vorzustellen. Prototyping durch einen vergleichsweisen geringen Arbeits- und Zeitaufwand sowie verhältnismäßig niedrige Kosten aus.

Um einen aussagekräftigen Prototyp zu erschaffen, gibt es im Design Thinking verschiedene Methoden. Hierbei wurde mein eigenes Wissen durch einen Artikel von Emprechtinger erweitert (\cite{prototyping_methoden}). Die Methoden werden nun vorgestellt und kurz bewertet: \\

\textbf{Skizzen und Zeichnungen:}	Diese Methode benötigt lediglich Papier und einige Stifte. Einfache Skizzen sollen helfen, die abstrakt scheinende Lösung zu veranschaulichen. Die entstandenen Illustrationen eignen sich gut zum Austausch im Team – gerade in frühen Iterationsphasen. Entstandene Skizzen und Mindmaps lassen sich als Grundlage für weitere Iterationen verwenden. Es gilt jedoch zu beachten, dass sich jene Teilnehmende, die mit dem Skizzieren oder Zeichnen nicht so vertraut sind, möglicherweise nicht hinreichend kreativ einbringen können.

\textbf{Paper Prototyping:} Diese Methode eignet sich zum Skizzieren digitaler Produkte wie Apps oder Webseiten. Hierbei werden die Mockups auf Papier gezeichnet und durch Zusammenkleben in einem Prototyp vereint. Ebenso lassen sich Webanwendungen auf dem Papier z.B. durch Umklappen von Papierfenstern erlebbar machen. Das Paper Prototype stellt analog zum einfachen Skizzieren eine leicht zu erlernende und kostengünstige Methode dar. Es genügen einige Blätter Papier sowie grundlegendes Bastelmaterial wie Kleber, Schere und Stifte.

\textbf{Rollenspiele:}	Hierbei steht die soziale Interaktion im Vordergrund. Sofern das Problem einen gesellschaftlichen Hintergrund hat, lässt sich die Lösung in einem Rollenspiel präsentieren. Im Zusammenspiel der Akteure werden die Probleme und deren Lösungsmöglichkeiten für die gesellschaftspolitischen Themen deutlich. Die Darstellung eines gesellschaftspolitischen Problems erfordert bei der Visualisierung mit einem Papiermodell eine starke Abstraktion. Dies ist bei einem Rollenspiel nicht der Fall. \newline
Die Methode benötigt prinzipiell keine Materialien; es empfiehlt sich jedoch, mit einigen Requisiten zu arbeiten. Um die Gefühlssituation bzw. den verkörperten Hintergrund des Schauspielers zu unterstützen, nutzt die HPI DSchool hierfür Perücken oder Kleidungsstücke.


\textbf{Videoclip:} Das Ziel der Methode ist das Vorstellen der Lösung in Form eines kurzen Videos. Diese Methode grenzt sich von den anderen dahingehend ab, dass der Film keiner weitere Erläuterung bedarf. Die entstandenen Clips können somit z. B. als Marketingvideo verwendet werden. Da das Endprodukt des Prototypings in diesem Fall digital vorliegt, kann das Video schnell und unkompliziert für mögliche Stakeholder zugänglich gemacht werden. In den Videos wird die Produktidee häufig mittels erklärender Skizzen erläutert. Videoclips werden aufgrund des hohen Erstellungsaufwandes häufig erst in den finalen Iterationen der Prototyping-Phase eingesetzt. Für das Erstellen aussagekräftiger medialer Prototypen sind die dafür benötigten Vorkenntnisse nicht zu unterschätzen. Das Konzipieren und Erstellen eines Videos gilt unter den Methoden als verhältnismäßig anspruchsvoll. 

\textbf{Lego Serious Play: } Beim LSP erhalten alle Teilnehmenden eine bestimmte Anzahl an Legosteinen. Aus diesem Bundle werden dann die Prototypen gebaut. Besonders häufig wird diese Methode in Workshops eingesetzt, in denen jeder Teilnehmer seinen eigenen Prototypen bauen soll. Sie eignet sich somit hervorragend für persönliche Leadership-Workshops wie z.B. Wayfinder. Eine detaillierte Vorstellung dieser Methode findet sich im Abschnitt \ref{intro_LSP}. \newline

Das Ziel der aufgeführen Prototyping-Methoden ist das möglichst schnelle und kostengünstige Visualisieren der Ideen. Zudem soll das Prototyping allen Teammitgliedern ermöglichen, aktiv am Visualisierungsprozess mitzuwirken. In den Phasen vor dem Prototyping werden Ideen häufig auf Postits gesammelt und geclustert. Hierbei wird eine ``grundlegende Kreativität'' benötigt, welche für die meisten Teilnehmenden kein Problem darstellt. Beim Prototyping hingegen werden Kreativität und künstlerische Fertigkeiten benötigt. Genau darin liegt die Stärke von LSP: Mit einheitlichen Bausteinen, deren Anordnung intuitiv ist, wird der Zugang zur Prototyping-Methode sehr leicht ermöglicht.

Betrachtet man abschließend die Phase des Prototypings mit deren Methoden, so gilt festzuhalten, dass es nicht erforderlich ist, sich auf eine Methode für alle Iterationen festzulegen. Gerade in den ersten Iterationen eignen sich zum Prototyping kurze Skizzen oftmals besser als fertige Videoclips. Diese Clips könnten in finalen Tests mit dem Endnutzer eingesetzt werden.


\pagebreak
\subsection{Einführung in  Lego Serious Play}
\label{intro_LSP}

Die Methode LSP wurde 1996 vom damaligen Geschäftsführer von LEGO\textsuperscript{\textregistered} Kjeld Kirk Kristiensen und zwei Professoren einer schweizer Universität entwickelt. Anfangs sollten mit der Methode vor allem Firmen erriecht werden. (\cite{lsp3D}) 2010 wurde LSP durch LEGO\textsuperscript{\textregistered} unter die Creative Commons Lizens gestellt. Seitdem hält LEGO\textsuperscript{\textregistered} keine Liste an zertifizierten LSP-Facilitatoren mehr. Ebenso entfällt die vorher erhobene Lizensgebühr für die zertifizierten Coaches. Ziel war es laut LEGO\textsuperscript{\textregistered}, die Methode einer breiteren Masse zugänglich zu machen. Zur Nutzung der Methode empfehle ich einen Blick in die englischsprachige Open-Source-Dokumentation\footnote{https://www.lego.com/en-us/seriousplay/background}.

Zum Anwenden der LSP-Methode benötigen die Teilnehmenden ein Set aus verschiedenen Legosteinen. Zudem empfiehlt sich ein weißes Blatt Papier als Unterlage, sowie ein Stift, um beim Erläutern des Modells auf bestimmte Teile zu verweisen. Das Set aus Legosteinen sollte nach Möglichkeit für alle Teilnehmenden identisch sein, damit sie dieselben Voraussetzungen haben. Sind Teile nur eingeschränkt verfügbar, können diese in einem ``Teilepool'' für alle zugänglich gemacht werden.

LSP zeichnet sich durch einen unkomplizierten Zugang aus, da viele Teilnehmenden mit dem Handling der Bausteine vertraut sind. Als Einstieg in die Methode ist es dennoch hilfreich, sich anfangs mit den gegebenen Steinen vertraut zu machen. Anfangs wird empfohlen, dass die Teilnehmenden die Bausteine genau betrachten und sortiert vor sich ablegen. In jedem Fall sollten vor der ersten konkreten Anwendung der Methode kleinere Übungen mit den Steinen gemacht werden, um alle Teilnehmenden in das richtige Mindset zu bringen. Ziel ist es, dass sich die Teilnehmenden mit den Bausteinen vertraut machen und über das Gebaute erstmalig reflektieren. Eine Aufgabe könnte sein, z. B.  eine möglichst lange Brücke oder einen möglichst hohen Turm zu bauen. (Aufgaben aus dem Wayfinder Seminar und anderen Ansätzen von \cite{aufwärmuebungen}) Hierbei sollte den Teilnehmenden ein kurzes Zeitfester zur Verfügung gestellt werden, um deren Kreativität anzuregen. Größere Zeitfenster sorgen oftmals für zu tiefgehende Gedanken über den Bauplan, was nicht das Ziel des Prototypings ist. Die beschränkte Anzahl an Steinen und Formen ist zudem Ansporn, Kreativität und Ideenvielfalt in das entstehende Modell einfließen zu lassen. Nach der Bauzeit finden sich die Teilnehmenden in Gruppen zusammen und stellen ihre Prototypen untereinander vor. Dabei können Rückfragen gestellt werden. Auch hierbei ist wieder auf enge Zeitplanung zu achten. Die Vorstellung der Modelle incl. Feedback sollte pro Person nie länger als 5 Minuten in Anspruch nehmen.

Im Anschluss daran ist es möglich, über das Modell zu iterieren. Dazu. kann es hilfreich sein, die Modelle zu zerlegen und hinsichtlich eines anderen Fokus komplett neu zu erbauen.
Ich habe persönlich gute Ergebnisse erzielt, wenn in der ersten Iteration die Modelle von jeder Einzelperson erstellt und reflektiert werden. In der nachfolgenden Iteration kreierte das Team dann aus den einzelnen Modellen einen gemeinsamen Prototyp.\newline

Entgegen der ersten Intuition können mittels LSP nicht nur materiell abbildbare Prototypen wie Stadtmodelle, Häuser oder andere ``physisch anfassbare Konzeptionen'' erstellt werden. Die Steine eignen sich ebenso hervorragend, um immaterielle Dinge wie Gefühlssituationen oder Lebensweisen darzustellen. Hierbei kommen vor allem Sondersteine wie Propeller, Pflanzen, Personen oder Fahnen zum Einsatz. Aber auch die Standardsteine (rechteckig und verschiedenfarbig) werden häufig verwendet, da sie durch die ``neutrale'' Form und Farbe keinen semantischen Impuls geben und somit einen hohen Interpretationsspielraum liefern. \newline

LSP wird jedoch nicht nur innerhalb von Design Thinking Prozessen genutzt. Aufgrund des einfachen Zugangs wird die Methode wird ebenso angewendet, um Strategieausrichtungen von Unternehmen zu modellieren oder Risikoszenarien zu prototypisieren.\newpage
	\section{Implementation of LSP in Wayfinder Seminar}

\subsection{Recap: How was LSP applied in WAYFINDER?}
Vielleicht geht das auch schon in 3b)

\subsection{Relation to WAYFINDER}
Why / How do you see this topic related to WAYFINDER?

\subsection{Integration into WAYFINDER}

\newpage
	\section{Diskussion}
\label{diskussion}

\subsection{Diskussion der Anwendung von LSP im Wayfinder}
Die Methode Lego Serious Play wurde für den digital stattfindenden Wayfinder-Workshop als wesentliche Prototyping-Methode ausgewählt. Diese Entscheidung ist aus meiner Sicht sehr überlegt getroffen worden.
Bei einer On-Site-Variante des Workshops hätten sich sicherlich zahlreiche Prototyping-Methoden angeboten. Aufgrund des remote Setups entfallen jedoch ziemlich alle der unter \ref{intro_prototyping} beschriebenen Methoden aufgrund des benötigten Materials. Für Skizzen oder Papier-Prototypen haben nicht alle Teilnehmenden das benötigte Kreativequipment zwangsweise vor Ort. In den Räumen der HPI DSchool gibt es hierfür gemeinsame Materialiensammlungen, sodass allen Teilnehmenden dieselben Mittel zur Verfügung stehen. Das kann bei einem nicht vor Ort stattfindenden Workshop nicht gewährleistet werden, weshalb diese Methoden eher nicht zu empfehlen sind. Rollenspiele eignen sich für individuelle Prototypen einer Person wie sie im Wayfinder angefertigt werden nicht so gut. Videoclips sind meinem Ermessen nach zu aufwändig für dieses Seminar.

Betrachten wir nun die aufgrund der speziellen Situation ausgewählte Methode. Bereits mit einem standardisierten Beutel an Legosteinen ist jeder Teilnehmende bereit für diesen Workshop. Diese Materialien bekamen wir postalisch zugesendet. Die weiße Unterlage war ebenso in dem Paket enthalten. Den zum Vorstellen benötigten Stift sollte jeder selbst beisteuern können. Somit hatten alle Teilnehmenden dieselben Ausgangsbedingungen für die Prototyping-Session.

Ein Nachteil von LSP bei On-Site Workshops kann aus meiner eigenen Erfahrung steigender Druck im Team sein. So lassen sich Teilnehmende während der Arbeit an ihrem individuellen Modell durch den Fortschritt der anderen oder den u. U. hohen Geräuschpegel während des Protoypings unter Druck setzen. Hierbei kam uns das remote-Format entgegen, welches dafür sorgte, dass alle Teilnehmenden akustisch ungestört und auch ohne den direkten Druck von Tischnachbarn an ihrem Modell arbeiten konnten.
Ebenso benötigt das Prototyping mit LSP nur sehr wenig Platz. Das Zusammenbauen des Modells mit maximal A4-Blatt-Größe ist auf dem heimischen Schreibtisch während einer Videokonferenz problemlos möglich gewesen. 

Um das Wayfinder-Seminar abwechslungsreicher zu gestalten, hätte man durchaus auch eine weitere Prototyping-Methode verwenden können. Hierfür wäre - wie oben beschrieben - lediglich das Zeichnen und Skizzieren geeignet gewesen. Aufgrund der benötigten ``grundlegenden Kreativität‘‘ für diese Methode, eignet sie sich bei unerfahrenem Teilnehmerfeld jedoch nicht so gut. Wenn Teilnehmer beim Prototypisieren nicht weiterkommen, kann ihnen bei remote Workshops nicht so schnell geholfen werden. Auf diese Hilfe ist man bei Methoden, die eine höhere Eigenkrativität erfordern, häufiger angewiesen.

Weitere Nachteile von Lego Serious Play sollen an dieser Stelle kritisch hinterfragt werden. Für einen zielführenden Workshop ist es aus meiner Sicht zwingende Voraussetzung, dass ausgebildete LSP ``Vermittler'' (Fachjargon: ``Facilitator'') durch den Workshop leiten. Ebenso sollten die betreuenden Coaches mit der Methode vertraut sein. Die benötigten Fortbildungsseminare sind jedoch recht kostenintensiv. Obwohl der  Zugang zur Methode sehr einfach ist, kann der spielerisch-kreative Ansatz von LSP bei einigen Teilnehmenden als nicht wissenschaftlich empfunden und damit die Methode unterschätzt werden. Selten sorgt auch der Name für einige Vorbehalte. Den Teilnehmenden mit dieser Grundeinstellung fällt der Zugang zur Methode vergleichsweise schwer. In unserem Workshop konnte ich das kaum beobachten. Das liegt aus meiner Sicht daran, dass uns die Coaches vor dem ersten Prototyping mittels kleiner Aufwärmübungen in das richtige Mindset gebracht haben, um mit der Methode zu arbeiten.

Erwähnt werden sollten noch die verhältnismäßig hohen Anschaffungskosten der LSP-Sets. Die so genannte ``Window Exploration Bag'' wurde in unserem Workshop verwendet. Diese kleinen Beutel eignen sich vor allem für kurze Workshops und kosten pro Person ca. 3.50 €\footnote{https://www.lego.com/de-de/product/window-exploration-bag-2000409}. Für längere Workshops eignen sich jedoch Sets mit verschiedenen Sonderbausteinen. Die hierfür vorgesehenen Sets von LEGO® kosten 24 € pro Person\footnote{https://www.lego.com/de-de/product/starter-kit-2000414}. 

Zudem erfordert die Anwendung von LSP bei Vor-Ort-Workshops eine besondere Aufmerksamkeit. Für ein entspanntes und kreatives Arbeiten sollte jedem Teilnehmenden ausreichend Arbeitsraum zur Verfügung gestellt werden, um konstruktiv an seinem Modell zu arbeiten. Der benötigte Platz ist somit größer als z. B. bei der Methode ``Sketching'' (Zeichnen). Darüber hinaus entsteht beim Durchsuchen von Lego-Boxen oder dem Umherschieben von Steinen ein nicht zu unterschätzender Geräuschpegel.

\pagebreak

\subsection{Fazit}

Die Methode Lego Serious Play ist sehr leicht und schnell erlernt. Gerade in remote Workshops eignet sie sich hervorragend zum Prototypisieren. Speziell auf dem Gebiet des individuellen Prototypings, wie es beim Wayfinder notwendig ist, ist sie meiner Meinung nach das Mittel der Wahl. Die oben genannten Nachteile von LSP wurden durch die Profis der HPI D-School gekonnt umgangen. Somit ergab sich ein sehr runder Workshop mit einem guten Exkurs in diese Methode!
	\newpage
	\singlespacing
	\pagenumbering{Roman}
	\setcounter{page}{\value{speichereRoemischeZahl}}
	\printbibliography[title={References}]
	\doublespacing
	\appendix % Anhang

    % Kapiteln mit Buchstaben beziffern
    %\renewcommand*\thechapter{\Alph{chapter}}
    \renewcommand{\thetable}{\Alph{table}}
    \setcounter{table}{0}
    \renewcommand{\thefigure}{\Alph{figure}}
    \setcounter{figure}{0}

\section{Anhang}

\begin{figure}[h]
    \includegraphics[width=\textwidth]{img/DT_process.png}
    \caption{The Design Thinking Process (\textcopyright ~HPI Academy)}
    \label{fig:dt_process}
\end{figure} 

\begin{figure}[h]
    \begin{center}
        \includegraphics[width=7.5cm]{img/week_2.jpg}
    \end{center}
    \caption{``My 2020'' - Prototype of week 2}
    \label{fig:week2}
\end{figure} 
\begin{center}
    \begin{figure}[h]
        \begin{center}
            \includegraphics[width=7.7cm]{img/week_3_1.jpg}
        \end{center}
        \caption{``Working remote'' - Prototype of week 3}
        \label{fig:week3}
    \end{figure} 
    \vspace{2 cm}
    \begin{figure}[h]
        \begin{center}
            \includegraphics[width=7.7cm]{img/week_4.jpg}
        \end{center}
        \caption{``The game plan'' - Prototype of week 4}
        \label{fig:week4}
    \end{figure} 
\end{center}


\pagebreak

\end{document}